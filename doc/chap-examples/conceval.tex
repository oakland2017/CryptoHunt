\section{Concrete Evaluation}

In this tutorial, we will explain how to raise a binary to the BAP IL,
and then evaluate the resulting code inside of BAP.

Let's start with the following simple C program:
\verbatiminput{chap-examples/test.c}

We can compile this source file (named test.c) to a binary using
\cmdline{gcc -static test.c -o test} (note if you are on an x86-64
machine you may need to add the \cmdline{-m32} option to follow along
with the example).  We use the \cmdline{-static} option of gcc here
because the BAP evaluator needs to know what code is located at each
memory location. This means that BAP does not currently support
self-modifying code, or dynamic linking (unless the user explicitly
tells BAP the mapping of memory addresses to binary code).

Once we have a binary, we can then raise it to the BAP IL by executing
\cmdline{toil -bin test -o test.il}. We can now disassemble the binary
to see where main begins and ends -- we will need this to tell BAP
where to start and stop executing. Here is a possible disassembly of
the main function.

\begin{verbatim}
080483cb <main>:
 80483cb:       55                      push   %ebp
 80483cc:       89 e5                   mov    %esp,%ebp
 80483ce:       83 ec 04                sub    $0x4,%esp
 80483d1:       c7 04 24 2a 00 00 00    movl   $0x2a,(%esp)
 80483d8:       e8 d7 ff ff ff          call   80483b4 <g>
 80483dd:       c9                      leave  
 80483de:       c3                      ret    
 80483df:       90                      nop
\end{verbatim}

For this sample disassembly, we can begin execution at address
\cmdline{0x80483cb}.  Likewise, we can halt execution at
\cmdline{0x80483de}.  To make BAP halt execution, we'll manually add a
halt command to the lifted IL.  For this example, you could search for
the string \verb!addr 0x80483de! in test.il, which marks the beginning
of the ret instruction in the lifted IL.  Inserting
\verb!halt R_EAX_32:u32!  after the address and label statements will
cause the evaluator to halt and return the value in eax.  For
instance:

\begin{verbatim}
addr 0x80483de @asm "ret    "
label pc_0x80483de
halt R_EAX_32:u32
\end{verbatim}

Now, issue \cmdline{ileval -il test.il -eval-at mainaddr}, where
mainaddr is the address to start from.  BAP will evaluate the IL
program.  If debugging output is enabled (by setting the
\texttt{BAP\_DEBUG\_MODULES} environment variable to
\texttt{Symbeval}), it will print each statement as it is evaluated.
After halting, BAP will print the return value, which is the final
value of \texttt{R\_EAX\_32}. Unsurprisingly, the returned value
should be 42.

Now let's begin execution starting from the call instruction in main
(\texttt{0x80483d8}). This will allow us to modify the input to the
function g. The \cmdline{-init-var} and \cmdline{-init-mem} options
allow the user to specify an initial value for a variable (register)
or memory address respectively.  To replicate what we just did and use
an input of 42, we can use the following the command \cmdline{ileval
  -il test.il -init-mem R\_ESP\_32:u32 42:u32 -eval-at
  0x80483d8}. Notice that this begins execution at the call
instruction instead of the beginning of main.  Unsurprisingly, the
returned value is 42 again.  If we instead change the input to 43, by
\cmdline{ileval -il test.il -init-mem R\_ESP\_32:u32 43:u32 -eval-at
  0x80483d8}, we obtain a final value of -1.

%%% Local Variables: 
%%% mode: latex
%%% TeX-master: "../main"
%%% End: 
