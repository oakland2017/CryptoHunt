\subsection{\bap Typing}
\label{vinesec:typecheck}

%\renewcommand{\baselinestretch}{1.0}\normalsize

\begin{table}

{\bf Context}\\
\begin{tabular}{llp{4.5in}}
$\Gamma$ & \emphkind{var} $\rightarrow$ $\tau$& 
Typing context of $(x:\tau)$ pairs where $x$ is of type $\tau$.
\end{tabular}\\[10pt]
{\bf Program and Instructions}
\begin{footnotesize}
\[
\begin{array}{c}
\infer[]
 { 
   \cdot \vdash (x_i:\tau_i)^* i^*  : ()
 }
 {
   \forall i \in i^* | (x_i:\tau_i)^* \vdash i : ()
 }
\end{array}
\]
\[
\begin{array}{ccc}
\infer[\textsc{assign}]
   {
     \Gamma \vdash x := e : ()
   }
   {
     \Gamma \vdash e : \tau
     &
     \Gamma \vdash x : \tau
   } &
\infer[\textsc{jmp}]
   {
     \Gamma \vdash \texttt{jmp($e$)} : ()
   }
   {
     \Gamma \vdash e : \texttt{reg64\_t}
   } & 
\infer[\textsc{halt}]
  {
     \Gamma \vdash \texttt{halt($e$)} : ()
  }
  { 
     \Gamma \vdash e : \tau
     & \tau \in \tau_{\text{reg}}
  }
\end{array}
\]
\[
\begin{array}{c}
\infer[\textsc{cjmp}]
   {
     \Gamma \vdash \texttt{cjmp($e_1$, $e_2$, $e_3$)} : ()
   }
   {
      \Gamma \vdash e_1 : \texttt{reg1\_t}
      & 
      \Gamma \vdash e_2 : \texttt{reg64\_t}
      &
      \Gamma \vdash e_3 : \texttt{reg64\_t}
   }
\end{array}
\]
\[
\begin{array}{ccc}
\infer[\textsc{assert}]
  {
    \Gamma \vdash \texttt{assert($e$)} : ()
  }
  {
    \Gamma \vdash e : \texttt{reg1\_t}
  } &
 \infer[\textsc{label}]
  {
    \Gamma \vdash \texttt{label $n$} : ()
  }
  {
  } &
\infer[\textsc{special}]
  {
   \Gamma \vdash \texttt{special {\sf id$_s$}} : ()
  }
  {
  }
\end{array}
\]
\end{footnotesize}
{\bf Auxiliary Definitions}
\begin{footnotesize}
\[
\begin{array}{c}
 \infer[\textsc{reg-subtype}]
   {
     \texttt{reg1\_t $<:$ reg8\_t $<:$ reg16\_t $<:$ reg32\_t $<:$ reg64\_t}
   }
   {
   }
\end{array}
\]
\[
\begin{array}{cc}
\infer[]
  {
    \text{mem\_compat($\tau_1$, $\tau_2$)}
  }
  {
   \tau_1 = \texttt{big $|$ little} 
    & \tau_2 \in \tau_{\text{reg}} 
  } &
\infer[]
  {
    \text{mem\_compat({\tt norm}, {\tt reg8\_t})}
  }
  {
  }\\[10pt]
\infer[]
  {
    \text{cast\_compat(k, $\tau_1$, $\tau_2$)}
  }
  {
    k = \texttt{unsigned $|$ signed}
    & 
    \tau_1 <: \tau_2
  }&
\infer[]
  {
    \text{cast\_compat(k, $\tau_1$, $\tau_2$)}
  }
  {
    k = \texttt{hi $|$ low}
    & 
    \tau_2 <: \tau_1
  }
\end{array}
\]
\end{footnotesize}
{\bf Expressions}
\begin{footnotesize}
\[
\begin{array}{cc}
\infer[\textsc{load}]
  { % conclusion
    \Gamma \vdash \texttt{load($e_1$, $e_2$, $\tau$)} : \tau
  }
  { % premise
    \Gamma \vdash e_1 : \texttt{mem\_t($\tau_1$, $\tau_2$)}
    & 
    \text{mem\_compat($\tau_1$, $\tau$)}
    &
    \Gamma \vdash e_2 : \tau_2
  } &
\infer[\textsc{var}]
  { % conclusion
      \Gamma \vdash x : \tau
  }
  { % premise
        x:\tau \in \Gamma
  }
\end{array}
\]
\[
\begin{array}{c}
\infer[\textsc{store}]
  { % conclusion
    \Gamma \vdash \texttt{store($e_1$, $e_2$,$e_3$)}:\texttt{mem\_t($\tau_1$, $\tau_2$)}
  }
  { %premise
    \Gamma \vdash e_1: \texttt{mem\_t($\tau_1$, $\tau_2$)}
    &
    \Gamma \vdash e_2 : \tau_2 
    & 
    \Gamma \vdash e_3 : \tau_4
    & 
    \text{mem\_compat($\tau_1$, $\tau_4$)}
  }\\
\end{array}
\]
\[
\begin{array}{cc}
\infer[\textsc{binop$_1$}]
  { % conclusion
    \Gamma \vdash e_1 \Diamond_b e_2 : \tau
  }
  { % premise
    \Gamma \vdash e_1 : \tau 
    & \Gamma \vdash e_2 : \tau
    & \tau \in \tau_{\text{reg}}
    & \Diamond_b \notin \ll, \gg, \gg_a
  } &
\infer[\textsc{unop}]
  { % conclusion
    \Gamma \vdash \Diamond_u e : \tau
  }
  { % premise
    \Gamma \vdash e : \tau 
    & \tau \in \tau_{\text{reg}}
  } 
\end{array}
\]
\[
\begin{array}{c}
\infer[\textsc{binop$_2$}]
  { % conclusion
    \Gamma \vdash e_1 \Diamond_b e_2 : \tau
  }
  { % premise
    \Gamma \vdash e_1 : \tau 
    & \Gamma \vdash e_2 : \tau_2
    & \tau, \tau_2 \in \tau_{\text{reg}}
    & \Diamond_b \in \ll, \gg, \gg_a
  } 
\end{array}
\]
\[
\begin{array}{cc}
\infer[\textsc{cast}]
  { % conclusion 
   \Gamma \vdash \texttt{cast(\emphkind{cast\_kind}, $\tau$, $e$)} : \tau
  }
  { % premise
    \Gamma \vdash e : \tau_1
    & \tau, \tau_1 \in \tau_{\text{reg}}
    & \text{cast\_compat(\emphkind{cast\_kind}, $\tau_1$, $\tau$)}
  } &
\infer[\textsc{let}]
  { % conclusion
    \Gamma \vdash \texttt{let $x$ := $e_1$ in $e_2$} : \tau
  }
  { % premise
    \Gamma \vdash e_1 : \tau_1
    & \Gamma,(x:\tau_1) \vdash e_2 : \tau
  } 

\end{array}
\]
\end{footnotesize}
\caption{Type-checking rules for \bap.}
\label{vine:typecheck}
\end{table}
%\renewcommand{\baselinestretch}{1.66}\normalsize


The \bap IL syntax described in Table~\ref{vine:syntax} is designed
to be simple. A literal interpretation, however, allows several
nonsensical operations such as left-shifting two memory variables.  We
type-check \bap programs using the  rules shown in
Table~\ref{vine:typecheck} to weed out obviously nonsensical
operations.  We emphasize that type-checking to show program safety is outside
the scope of \bap.

The type-checking rules are straight-forward.  \bap requires that
binary and unary operations be on scalars of the same register type
(except shifts where the shift amount need only be an integer). We
type-check {\tt hi} and {\tt low} casts such that the resulting type
is a smaller register type (in terms of the sub-typing relationship
{\sc reg-subtype}), as expected.  {\tt unsigned} and {\tt signed}
casts must be to a wider register type. One issue with un-normalized
memory is that it may have multi-byte accesses, e.g., a store on x86
could be of 32, 16, or 8 bits. Normalized memory, on the other hand,
only allows stores and loads of bytes.  We define an auxiliary
predicate ``mem\_compat'' to ensure that normalized memory is accessed
appropriately.  Also note that as a convenience we assume 64-bit
addressing in the rules, but allow smaller addressable memories via
the {\tt reg-subtype} sub-typing rule.

A program type-checks if all instructions type-check under the
declared variable types.  We require all variables be declared because
explicitly typed languages are easier to check. 

